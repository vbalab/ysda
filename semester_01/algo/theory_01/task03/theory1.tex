\newcommand{\groupnumber}{1}

\documentclass[11pt]{article}

\usepackage{hometask}
\usepackage{amsfonts}
\usepackage{expdlist}
\usepackage{cmap}  % should be before fontenc
\usepackage[T2A]{fontenc}
\usepackage[utf8]{inputenc}
\usepackage[russian]{babel}
\usepackage{amsmath}
\usepackage{amssymb}
\usepackage{amsthm}

\newcommand{\E}{\ensuremath{\mathsf{E}}}  % матожидание
\newcommand{\D}{\ensuremath{\mathsf{D}}}  % дисперсия
\newcommand{\Prb}{\ensuremath{\mathsf{P}}}  % вероятностная мера

\newcommand{\eps}{\varepsilon}  % эпсилон
\renewcommand{\phi}{\varphi}    % фи

\renewcommand{\le}{\leqslant}   % <=
\renewcommand{\leq}{\leqslant}  % <=
\renewcommand{\ge}{\geqslant}   % >=
\renewcommand{\geq}{\geqslant}  % >=

\begin{document}

\begin{problem}{15}
Рассмотрим задачу слияния двух упорядоченных последовательностей длины $m$ и $n$, $m \ge 2n$. Докажите (используя модель решающего дерева)
нижнюю оценку $\Omega(n \log \frac{m}{n})$ на количество сравнений, необходимых для решения данной задачи в худшем случае.
\end{problem}

\begin{solution}
	При слиянии массивов нам нужно выбрать $ n $ позиций из $ m + n $ для элементов из $ B $, так как остальные позиции будут заняты элементами из $ A $. Число способов выбрать $ n $ позиций из $ m + n $ равно $ C^{n}_{ n + m} $, что равно количеству различных способов размещения элементов A и B в результирующей последовательности, а соответственно, минимально возможное количество листьев корректного дерева.

	Из двоичного ветвления решающего дерева следует, что глубина дерева не меньше $ \log(C^{n}_{ n + m}) $. Имея ввиду то, что $ C^{n}_{ n + m} \approx \frac{(n + m)^n}{n^n} $ по формуле Стирлинга, получаем, что
	$$
		\Omega(\log(C^{n}_{ n + m})) = \Omega(\frac{(n + m)^n}{n^n}) = \Omega((\frac{m}{n})^n) = \Omega(n \log \frac{m}{n}).
	$$
\end{solution}

\end{document}
