\newcommand{\groupnumber}{1}

\documentclass[11pt]{article}

\usepackage{hometask}
\usepackage{amsfonts}
\usepackage{expdlist}
\usepackage{cmap}  % should be before fontenc
\usepackage[T2A]{fontenc}
\usepackage[utf8]{inputenc}
\usepackage[russian]{babel}
\usepackage{amsmath}
\usepackage{amssymb}
\usepackage{amsthm}

\newcommand{\E}{\ensuremath{\mathsf{E}}}  % матожидание
\newcommand{\D}{\ensuremath{\mathsf{D}}}  % дисперсия
\newcommand{\Prb}{\ensuremath{\mathsf{P}}}  % вероятностная мера

\newcommand{\eps}{\varepsilon}  % эпсилон
\renewcommand{\phi}{\varphi}    % фи

\renewcommand{\le}{\leqslant}   % <=
\renewcommand{\leq}{\leqslant}  % <=
\renewcommand{\ge}{\geqslant}   % >=
\renewcommand{\geq}{\geqslant}  % >=

\begin{document}

\begin{problem}{20}
	Предложите реализацию \emph{cтека} на основе (одного) массива, которая поддерживает операции добавления в конец и удаления из конца. Требуется, чтобы \emph{емкость} (количество выделенных ячеек памяти) стека в любой момент времени отличалась от фактического размера не более чем в константу раз, а учетная сложность операций добавления в конец и удаления из конца была константной.
	\end{problem}
	
	\begin{solution}
		Будет использоваться динамический массив с изменением размера (увеличением/уменьшением) размера в константу раз (напр., в 2 раза) при достижении \texttt{size}'а \texttt{capacity} или при уменьшении \texttt{size}'а в константу раз (минус единица) от \texttt{capacity}.
	
		Это позволяет обеспечить, чтобы емкость стека была не более чем в константу раз больше его фактического размера и чтобы учетная сложность операций добавления и удаления была $O(1)$. При этом, стоимость операций texttt{push} и \texttt{pop} имеют амортизированную сложность $O(1)$.
	
		Для доказательства амортизированной стоимости операций \texttt{push} и \texttt{pop} в стеке с динамическим изменением размера используем \textbf{метод банкира} (accounting method). Каждой операции присваивается амортизированная стоимость \(3\) единицы, что покрывает её фактическую стоимость и накапливает "кредит" для более затратных операций изменения размера.
	
		Если при добавлении (\texttt{push}) массив не заполнен, фактическая стоимость составляет \(1\), а оставшиеся \(2\) единицы идут в "кредит". Если массив заполнен, емкость удваивается и элементы копируются, что стоит \(n\) операций (где \(n\) — текущий размер). Однако накопленные \(2n\) единиц кредита с предыдущих \texttt{push} покрывают затраты на удвоение.
	
		Аналогично, при удалении (\texttt{pop}) амортизированная стоимость \(3\) покрывает \(1\) единицу на удаление и оставляет \(2\) в "кредит". Если размер уменьшается до \(\text{capacity}/4\), емкость делится пополам и требуется \(n\) операций для копирования, что компенсируется накопленным кредитом \(2n\).
	
		Таким образом, амортизированная стоимость операций \texttt{push} и \texttt{pop} — \(3\), обеспечивая амортизированную сложность \(O(1)\).
	
	\end{solution}
	
\end{document}