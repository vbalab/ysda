\newcommand{\groupnumber}{1}

\documentclass[11pt]{article}

\usepackage{hometask}
\usepackage{amsfonts}
\usepackage{expdlist}
\usepackage{cmap}  % should be before fontenc
\usepackage[T2A]{fontenc}
\usepackage[utf8]{inputenc}
\usepackage[russian]{babel}
\usepackage{amsmath}
\usepackage{amssymb}
\usepackage{amsthm}

\newcommand{\E}{\ensuremath{\mathsf{E}}}  % матожидание
\newcommand{\D}{\ensuremath{\mathsf{D}}}  % дисперсия
\newcommand{\Prb}{\ensuremath{\mathsf{P}}}  % вероятностная мера

\newcommand{\eps}{\varepsilon}  % эпсилон
\renewcommand{\phi}{\varphi}    % фи

\renewcommand{\le}{\leqslant}   % <=
\renewcommand{\leq}{\leqslant}  % <=
\renewcommand{\ge}{\geqslant}   % >=
\renewcommand{\geq}{\geqslant}  % >=

\begin{document}

\begin{problem}{30}
	Рассмотрим алгоритмическую задачу с конечным множеством входов $\mathcal{I}$ и произвольное вероятностное распределение на нем. Рассмотрим также некоторое конечное семейство $\mathcal{A}$ детерминированных алгоритмов для ее решения
	и вероятностное распределение на нем. Обозначим через $f_A(I)$ время работы алгоритма
	$A$ на входе $I$.
	
	\begin{enumerate}
		\item Докажите неравенство
			  $$
				  \min_{A \in \mathcal{A}} E_I\left[ f_A(I) \right] \le \max_{I \in \mathcal{I}} E_A\left[ f_A(I) \right],
			  $$
			  где матожидания взяты относительно соответствующих распределений.
			  Сформулируйте данное неравенство в терминах соотношения сложности в среднем (по входам) и рандомизированной сложности.
	
			  \begin{solution}
				  В силу дискретности распределений величин $I \in \mathcal{I} $ и $A \in \mathcal{A} $ получаем, что $ E_I(f_A(I)) = \sum_{I \in \mathcal{I}} P(I) f_A(I) $, а также $ E_A(f_A(I)) = \sum_{A \in \mathcal{A}} Q(A) f_A(I) $, с соответствующими таблицами распределения $ P $ и $ Q $.
	
				  Значит, требуемое неравенство следует из очевидной цепочки сравнений:
				  $$
					  \min_{A \in \mathcal{A}} E_I\left[ f_A(I) \right] = \min_{A \in \mathcal{A}} \sum_{I \in \mathcal{I}} P(I) f_A(I) \le \sum_{A \in \mathcal{A}} \sum_{I \in \mathcal{I}} P(I) Q(A) f_A(I) = E_{I, A}(f_A(I)),
				  $$
				  а также
				  $$
					  E_{I, A}(f_A(I)) = \sum_{I \in \mathcal{I}} \sum_{A \in \mathcal{A}} P(I) Q(A) f_A(I) \le \max_{I \in \mathcal{I}} \sum_{A \in \mathcal{A}} Q(A) f_A(I) = \max_{I \in \mathcal{I}} E_A\left[ f_A(I) \right].
				  $$
	
				  Само же неравенство можно интерпретировать как то, что время, затрачиваемое лучшим в среднем (по входу) алгоритмом, всегда не больше, чем время, затрачиваемое на наихудший вход некоторого алгоритма "среднячка".
	
				  Это показывает связь между детерминированными алгоритмами с минимальной средней сложностью и рандомизированными алгоритмами с максимальной ожидаемой сложностью по всем входам.
			  \end{solution}
	
		\item Придумайте и сформулируйте определение \emph{рандомизированного} алгоритма сортировки в модели решающих деревьев.
		Докажите в этой модели оценку $\Omega(n \log n)$ для сложности произвольного рандомизированного      алгоритма сортировки, основанного на сравнении ключей.
	
			  \begin{solution}
				  Рандомизированность алгоритма сортировки в модели решающих деревьев можно интерпретировать как выбор случайных элементов сравнения за место детерминированных индексов элементов.
	
				  В ванильном детерминированном алгоритме сортировки в модели решающих деревьев из-за $ n! $ количества перестановок и необходимости (для корректности) наличия соответствующих (как минимум) $ n! $ листьев, получаем $ \log_2 n! $ глубину дерева, которая асимптотически (по формуле Стирлинга) равна $ n \log n $.
	
				  Рандомизированный алгоритм может давать разные деревья на одном и том же входе в зависимости от draw случайного числа, но ожидаемая глубина его выполнения всё равно ограничена снизу $\Omega(n \log n)$, поскольку даже при наличии случайности, алгоритм должен учитывать все возможные перестановки входов, что требует по меньшей мере $ \log_2 (n!) $ шагов.
			  \end{solution}
	\end{enumerate}
	\end{problem}
	
\end{document}