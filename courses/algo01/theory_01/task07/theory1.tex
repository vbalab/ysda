\newcommand{\groupnumber}{1}

\documentclass[11pt]{article}

\usepackage{hometask}
\usepackage{amsfonts}
\usepackage{expdlist}
\usepackage{cmap}  % should be before fontenc
\usepackage[T2A]{fontenc}
\usepackage[utf8]{inputenc}
\usepackage[russian]{babel}
\usepackage{amsmath}
\usepackage{amssymb}
\usepackage{amsthm}

\newcommand{\E}{\ensuremath{\mathsf{E}}}  % матожидание
\newcommand{\D}{\ensuremath{\mathsf{D}}}  % дисперсия
\newcommand{\Prb}{\ensuremath{\mathsf{P}}}  % вероятностная мера

\newcommand{\eps}{\varepsilon}  % эпсилон
\renewcommand{\phi}{\varphi}    % фи

\renewcommand{\le}{\leqslant}   % <=
\renewcommand{\leq}{\leqslant}  % <=
\renewcommand{\ge}{\geqslant}   % >=
\renewcommand{\geq}{\geqslant}  % >=

\begin{document}

\begin{problem}{30}
	Покажите, как \emph{деамортизировать} операции вставки в конец вектора, т.е. добиться того, чтобы операции добавления в конец и чтения элемента по индексу требовали $O(1)$ времени в \emph{худшем} случае. Совет: по мере добавления новых элементов необходимо параллельно копировать уже имеющийся массив в массив увеличенного размера. Делать это следует с такой скоростью, чтобы в тот момент, когда меньший массив окажется заполнен, мы могли за время $O(1)$ выполнить переключение на новый массив.
	\end{problem}
	
	\begin{solution}
		Для деамортизации операций вставки в конец динамического массива и обеспечения того, чтобы добавление элемента и чтение по индексу имели сложность $O(1)$ в худшем случае, используем стратегию параллельного копирования.
	
		\begin{enumerate}
			\item Процесс начинается с аллокации 2х массивов с константными длинами $ k $ и $ k * 2 $, где $ k $ >= 2 и $ k $ - четное. Обозначим меньший массив за $S$, а больший за $L$.
			
			\item В каждый момент времени имеем два массива с константной \texttt{capacity} $ n $ и $ 2n $ у $S$ и $L$ соответственно. Пользователь взаимодействует с $S$. Пока \texttt{size}$_S$ $<$ половины \texttt{capacity}$_S$, делаем \texttt{push} лишь в $S$.
	
			\item Как только \texttt{size}$_S$ достигает половины \texttt{capacity}$_S$, делаем такой же \texttt{push} в $S$. Создаем \texttt{index}$=0$ на элементы массива $S$ и делаем их \texttt{push} в массив $L$ дважды, увеличивая \texttt{index}.
		
			\item Когда $S$ достигает своей \texttt{capacity}$_S$, "переключаем" пользователя на $L$, который к этому моменту имеет все элементы $S$ и достиг половины \texttt{capacity}$_L$. Теперь $S:=L$, старый $S$ деаллоцируется и новый L с удвоенной \texttt{capacity}$_S$ аллоцируется. Здесь считаем, что операции аллокации и деаллокации памяти константны.
		\end{enumerate}
	
		Благодаря этому процессу переключение между старым и новым массивом происходит плавно и не требует $O(n)$ времени в момент переключения.
		Таким образом, использование параллельного копирования позволяет деамортизировать вставку в конец вектора, обеспечивая гарантированное время $O(1)$ на каждую операцию добавления, даже в момент изменения емкости массива.
	\end{solution}
	
\end{document}