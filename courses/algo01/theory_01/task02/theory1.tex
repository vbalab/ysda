\newcommand{\groupnumber}{1}

\documentclass[11pt]{article}

\usepackage{hometask}
\usepackage{amsfonts}
\usepackage{expdlist}
\usepackage{cmap}  % should be before fontenc
\usepackage[T2A]{fontenc}
\usepackage[utf8]{inputenc}
\usepackage[russian]{babel}
\usepackage{amsmath}
\usepackage{amssymb}
\usepackage{amsthm}

\newcommand{\E}{\ensuremath{\mathsf{E}}}  % матожидание
\newcommand{\D}{\ensuremath{\mathsf{D}}}  % дисперсия
\newcommand{\Prb}{\ensuremath{\mathsf{P}}}  % вероятностная мера

\newcommand{\eps}{\varepsilon}  % эпсилон
\renewcommand{\phi}{\varphi}    % фи

\renewcommand{\le}{\leqslant}   % <=
\renewcommand{\leq}{\leqslant}  % <=
\renewcommand{\ge}{\geqslant}   % >=
\renewcommand{\geq}{\geqslant}  % >=

\begin{document}
\begin{problem}{15}
Рассмотрим произвольное (корректное) решающее дерево для задачи сортировки $n$ ключей, в котором все листья являются достижимыми.
Найдите точную (не асимптотическую!) нижнюю оценку на \emph{наименьшую} из глубин листьев данного дерева, т.е.
такую функцию $g(n)$, что в любом подобном дереве наименьшая глубина листа не меньше $g(n)$,
и одновременно для любого $n$ существует такое дерево, в котором наименьшая глубина листа равна $g(n)$.
\end{problem}

\begin{solution}
	Эта глубина соответствует минимальному числу сравнений, необходимых для сортировки ключей в наилучшем случае.

	Каждое сравнение может определить порядок между двумя элементами, и для установления полного порядка среди $ n $ элементов необходимо как минимум $ n - 1 $ парных сравнений (то есть, нельзя меньше), что выступает глубиной минимального листа.

	При этом, для любого набора из $ n  $ ключей всегда найдется такое дерево, в котором существует путь, в котором каждое сравнение ведет в одну и ту же сторону (т.е., $ \forall a,b \in \{a_1, ..., a_n\} $ сравнение $ a < b $ либо только удовлетворяется, либо нет), при этом, сравнения не повторяются. Поэтому, для достижения отсортированного массива из $ n $ ключей выполняется $ n - 1 $ сравнений.

	В результате, $ g(n) = n - 1 $; то есть минимальная глубина листа не меньше $ n - 1 $ и может быть равна $ n - 1 $.

\end{solution}

\end{document}
