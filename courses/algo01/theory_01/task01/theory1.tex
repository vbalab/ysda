\newcommand{\groupnumber}{1}

\documentclass[11pt]{article}

\usepackage{hometask}
\usepackage{amsfonts}
\usepackage{expdlist}
\usepackage{cmap}  % should be before fontenc
\usepackage[T2A]{fontenc}
\usepackage[utf8]{inputenc}
\usepackage[russian]{babel}
\usepackage{amsmath}
\usepackage{amssymb}
\usepackage{amsthm}

\newcommand{\E}{\ensuremath{\mathsf{E}}}  % матожидание
\newcommand{\D}{\ensuremath{\mathsf{D}}}  % дисперсия
\newcommand{\Prb}{\ensuremath{\mathsf{P}}}  % вероятностная мера

\newcommand{\eps}{\varepsilon}  % эпсилон
\renewcommand{\phi}{\varphi}    % фи

\renewcommand{\le}{\leqslant}   % <=
\renewcommand{\leq}{\leqslant}  % <=
\renewcommand{\ge}{\geqslant}   % >=
\renewcommand{\geq}{\geqslant}  % >=

\begin{document}

\begin{problem}{5}
Пусть $t_r(I)$ обозначает время рандомизированного алгоритма на входе $I$ при значениях случайных бит $r$. Для заданного множества входов $\mathcal{I}$
рассмотрим две ``меры сложности``:
$$
	A = \mathop{\rm max}_{I \in \mathcal{I}} E_r \left[ t_r(I) \right], \qquad
	B = E_r \left[ \mathop{\rm max}_{I \in \mathcal{I}} t_r(I) \right].
$$
Можно ли утверждать, что одно из чисел $A$ и $B$ всегда не меньше другого?
\end{problem}

\begin{solution}
	Используя неравенство Йенсена, можно показать, что $B \ge A$, что следует из вогнутости функции $\max_{I \in \mathcal{I}} t_r(I)$ по случайным битам $r$. Вогнутость функции $\max$ в данном контексте следует из свойств функций максимума в пространстве случайных величин.

	Поэтому ожидание максимума всегда не меньше максимума ожиданий.
\end{solution}

\end{document}
