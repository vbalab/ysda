\newcommand{\groupnumber}{1}

\documentclass[11pt]{article}

\usepackage{hometask}
\usepackage{amsfonts}
\usepackage{expdlist}
\usepackage{cmap}  % should be before fontenc
\usepackage[T2A]{fontenc}
\usepackage[utf8]{inputenc}
\usepackage[russian]{babel}
\usepackage{amsmath}
\usepackage{amssymb}
\usepackage{amsthm}

\newcommand{\E}{\ensuremath{\mathsf{E}}}  % матожидание
\newcommand{\D}{\ensuremath{\mathsf{D}}}  % дисперсия
\newcommand{\Prb}{\ensuremath{\mathsf{P}}}  % вероятностная мера

\newcommand{\eps}{\varepsilon}  % эпсилон
\renewcommand{\phi}{\varphi}    % фи

\renewcommand{\le}{\leqslant}   % <=
\renewcommand{\leq}{\leqslant}  % <=
\renewcommand{\ge}{\geqslant}   % >=
\renewcommand{\geq}{\geqslant}  % >=

\begin{document}

\begin{problem}{30}
Семейство $\mathcal{H}$ хеш-функций $h$, принимающих $m$ значений, называется \emph{$C$-универсальным}, если для произвольных различных ключей $x$ и $y$ справедливо $P[h(x)=h(h)] \le C/m$.
Рассмотрим семейство хеш-функций
$$
    h_{a}(x) = \lfloor \left( (ax) \bmod 2^w \right) / 2^{w-l}\rfloor,
$$
определенных на $w$-битных ключах $x$ ($0 \le x < 2^w$) и дающих $l$-битные хеш-значения,
где $a$~--- случайное нечетое число на отрезке $[1,2^w]$.
Покажите, что существует абсолютная константа $C$, такая что данное семейство является $C$-универсальным.
\end{problem}

\begin{solution}
\textbf{Обозначения:}

\begin{itemize}
    \item $w$ — количество бит в ключах $x$, $0 \le x < 2^w$.
    \item $l$ — количество бит в хеш-значениях, $m = 2^l$.
    \item $a$ — случайное нечетное число из множества $\{1, 3, 5, \dots, 2^w - 1\}$ (всего $2^{w - 1}$ чисел).
\end{itemize}

\textbf{Идея доказательства:}

Мы будем анализировать вероятность совпадения хеш-значений $h_{a}(x)$ и $h_{a}(y)$ при случайном выборе $a$. Для этого рассмотрим разность $s_x - s_y$, где $s_x = (a x) \bmod 2^w$ и $s_y = (a y) \bmod 2^w$. Заметим, что $s_x - s_y \equiv a (x - y) \pmod{2^w}$.

Совпадение хеш-значений $h_{a}(x) = h_{a}(y)$ эквивалентно тому, что верхние $l$ битов $s_x$ и $s_y$ совпадают, то есть разность $s_x - s_y$ делится на $2^{w - l}$.

\textbf{Доказательство:}

\begin{enumerate}
    \item \textbf{Определим разность $d = x - y \not= 0$.}

          Поскольку $x \not= y$, то $d \not= 0$ и $d \in \{-2^w + 1, \dots, 2^w - 1\}$.

    \item \textbf{Выразим условие совпадения хеш-значений через $a$ и $d$.}

          Хеш-значения совпадают, если
          \[
              h_{a}(x) = h_{a}(y) \quad \Leftrightarrow \quad \left\lfloor \dfrac{(a x) \bmod 2^w}{2^{w - l}} \right\rfloor = \left\lfloor \dfrac{(a y) \bmod 2^w}{2^{w - l}} \right\rfloor.
          \]
          Это эквивалентно тому, что
          \[
              \dfrac{(a x) \bmod 2^w}{2^{w - l}} - \dfrac{(a y) \bmod 2^w}{2^{w - l}} \in [0,1).
          \]
          Таким образом,
          \[
              ((a x) - (a y)) \bmod 2^w < 2^{w - l}.
          \]
    \item \textbf{Преобразуем неравенство:}

          \[
              (a d) \bmod 2^w < 2^{w - l}.
          \]
          Это означает, что $a d$ делится на $2^{w - l}$ при приведении по модулю $2^w$, то есть
          \[
              2^{w - l} \mid (a d) \pmod{2^w}.
          \]

    \item \textbf{Рассмотрим два случая в зависимости от $d$.}

          \textbf{Случай 1:} $d$ чётное, то есть $v_2(d) \ge 1$, где $v_2(d)$ — показатель степени при разложении $d$ на простые множители, соответствующий двойке.

          Пусть $v_2(d) = t \ge 1$, тогда $d = 2^t d'$, где $d'$ — нечётное число.

          \textbf{Случай 2:} $d$ нечётное, то есть $v_2(d) = 0$.

    \item \textbf{Анализируем вероятность в обоих случаях.}

          \textbf{Случай 1:} $v_2(d) = t \ge 1$.

          Поскольку $a$ — нечётное, $v_2(a) = 0$. Тогда $v_2(a d) = v_2(a) + v_2(d) = t$.

          Значит, $a d$ делится на $2^t$, но не на $2^{t + 1}$.

          Для того чтобы $2^{w - l} \mid a d$, необходимо, чтобы $w - l \le t$, то есть $t \ge w - l$.

          Если $t \ge w - l$, то $v_2(a d) \ge w - l$, и неравенство выполняется.

          В этом случае $a d \bmod 2^w$ будет делиться на $2^{w - l}$.

          \textbf{Случай 2:} $v_2(d) = 0$.

          Поскольку $v_2(a) = 0$, то $v_2(a d) = 0$. Следовательно, $a d$ не делится на $2^{w - l}$, и неравенство не выполняется.

    \item \textbf{Вывод вероятности:}

          \textbf{Случай 1:} Когда $v_2(d) \ge w - l$, то $a d \bmod 2^w$ делится на $2^{w - l}$.

          Однако, поскольку $v_2(a d) = t$, а $t \ge w - l$, то
          \[
              a d \bmod 2^w = 2^{t} k \pmod{2^w},
          \]
          где $k$ — нечётное число.

          Количество таких $a$ равно количеству нечётных чисел в $[1, 2^w - 1]$, то есть $2^{w - 1}$.

          Но поскольку $d$ фиксировано, и $a$ пробегает все нечётные числа, $a d \bmod 2^w$ будет принимать каждое значение, кратное $2^{t}$, ровно один раз.

          Число значений $a$, для которых $a d \bmod 2^w$ делится на $2^{w - l}$, равно
          \[
              N = \dfrac{2^{w - 1}}{2^{t - (w - l)}} = 2^{l - 1}.
          \]
          Здесь мы делим общее число нечётных $a$ на количество возможных значений $a d \bmod 2^w$, которые делятся на $2^{w - l}$.

          Таким образом, вероятность
          \[
              \Prb[h_{a}(x) = h_{a}(y)] = \dfrac{N}{2^{w - 1}} = \dfrac{2^{l - 1}}{2^{w - 1}} = \dfrac{1}{2^{w - l}} = \dfrac{1}{2^{l}} = \dfrac{1}{m}.
          \]

          \textbf{Случай 2:} Когда $v_2(d) < w - l$.

          Тогда $a d \bmod 2^w$ не делится на $2^{w - l}$, и неравенство не выполняется для любого $a$.

          Таким образом,
          \[
              \Prb[h_{a}(x) = h_{a}(y)] = 0 \le \dfrac{1}{m}.
          \]
\end{enumerate}
\end{solution}

\end{document}