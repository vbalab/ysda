\documentclass{article}

\usepackage[a4paper, margin=0.75in]{geometry} % Adjust margin here
\usepackage{cmap}  % should be before fontenc
\usepackage[T2A]{fontenc}
\usepackage[utf8]{inputenc}
\usepackage[russian]{babel}
\usepackage{amsmath}
\usepackage{amssymb}
\usepackage{amsthm}
\usepackage[pdftex,colorlinks=true,linkcolor=blue,urlcolor=red,unicode=true,hyperfootnotes=false,bookmarksnumbered]{hyperref}
\usepackage{indentfirst}

\newtheorem{lemma}{Лемма}  % создаёт команд для лемм
\newtheorem{theorem}{Теорема}  % создаёт команд для теорем

\newcommand{\E}{\ensuremath{\mathsf{E}}}  % матожидание
\newcommand{\D}{\ensuremath{\mathsf{D}}}  % дисперсия
\newcommand{\Prb}{\ensuremath{\mathsf{P}}}  % вероятностная мера

\newcommand{\eps}{\varepsilon}  % эпсилон
\renewcommand{\phi}{\varphi}    % фи

\renewcommand{\le}{\leqslant}   % <=
\renewcommand{\leq}{\leqslant}  % <=
\renewcommand{\ge}{\geqslant}   % >=
\renewcommand{\geq}{\geqslant}  % >=

\pagestyle{myheadings}
\markright{Владислав Балабаев, vbalab\hfill}

\begin{document}

\section{Алгоритм решения}

Обозначим через $V$ множество значений количества слитков у людей, где $v_i$ — количество слитков у человека $i$, где $v_i \in [1, \bar{V}]$.

\begin{enumerate}
    \item Инициализируем границы бинарного поиска:
    \[
    \texttt{left\_ptr} = 0,\quad \texttt{right\_ptr} = \max\{v_i : v_i \in V\} + 1
    \]
    
    \item Повторяем, пока $\texttt{right\_ptr} - \texttt{left\_ptr} > 1$:
    \begin{itemize}
        \item Вычисляем $\texttt{mid} = \left\lfloor \frac{\texttt{left\_ptr} + \texttt{right\_ptr}}{2} \right\rfloor$;
        \item Проверяем возможность перераспределения с ограничением не более чем \texttt{mid} слитков у одного человека;
        \item Если возможно — обновляем \texttt{right\_ptr}, иначе — \texttt{left\_ptr}.
    \end{itemize}
    
    \item Ответ — минимальное допустимое значение \texttt{right\_ptr}.
\end{enumerate}

\section{Проверка перераспределения через поток}

Построим сеть:
\begin{itemize}
    \item Исток $s$ соединяется со всеми вершинами (людьми) — пропускная способность $v_i$;
    \item Из каждой вершины:
    \begin{enumerate}
        \item В другие вершины по графу доверия — с пропускной способностью $v_i$;
        \item В сток $t$ — с пропускной способностью, равной \texttt{mid}.
    \end{enumerate}
\end{itemize}

Алгоритм поиска потока — Эдмондса-Карпа.

\section{Корректность}

\begin{lemma}
    Функция проверки возможности перераспределения монотонна по \texttt{mid}.
\end{lemma}

\begin{proof}
    При увеличении \texttt{mid} возрастает пропускная способность рёбер в сток, что увеличивает или сохраняет величину максимального потока. Следовательно, если перераспределение невозможно при $k$, то оно также невозможно при всех $k' < k$.
\end{proof}

\begin{theorem}
    Алгоритм бинарного поиска по ответу с использованием поиска потока корректен.
\end{theorem}

\begin{proof}
    Следует из монотонности функции проверки и корректности алгоритма Эдмондса-Карпа для нахождения максимального потока.
\end{proof}

\section{Временная сложность}

\begin{itemize}
    \item Бинарный поиск: $O(\log \bar{V})$;
    \item Поиск потока Эдмондса-Карпа: $O(VE^2)$;
\end{itemize}

Общая сложность: $O(\log \bar{V} \cdot VE^2)$.

\section{Затраты по памяти}

Хранение графа требует $O(V + E)$ памяти. Остальные расходы пренебрежимо малы.

\end{document}